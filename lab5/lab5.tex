\section{Lab5: Modulación BPSK en GRC}

%*********************
\begin{frame}{}

\pgfdeclareimage[width=\paperwidth,height=\paperheight]{bg}{imagenes/fondocap2}
\setbeamertemplate{background}{\pgfuseimage{bg}}

\bfseries{\textrm{\LARGE Lab5\\ \Large Modulación BPSK en GRC}}
\raggedright
\end{frame}
%*********************


\begin{frame}{Modulación BPSK en GRC}

\pgfdeclareimage[width=\paperwidth,height=\paperheight]{bg}{imagenes/fondo3}
\setbeamertemplate{background}{\pgfuseimage{bg}}


\begin{figure}
  \centering
   \includegraphics[width=\textwidth]{lab5/pdf/151.pdf}
  \end{figure}
Convierte el flujo de datos digital en señal analógica de banda base (muestreada) utilizando el filtro FIR de interpolación.
\end{frame}

\begin{frame}{Modulación BPSK en GRC}
\begin{figure}[H]
\centering
\includegraphics[width=\textwidth]{lab5/pdf/152.pdf}
\end{figure}
\end{frame}

\begin{frame}{Modulación BPSK en GRC}
\begin{figure}
  \centering
   \includegraphics[width=\textwidth]{lab5/pdf/153.pdf}
  \end{figure}
  \centering
Modulación BPSK en función del tiempo
\end{frame}

\begin{frame}{Modulación BPSK en GRC}
\begin{figure}
  \centering
   \includegraphics[width=0.8\textwidth]{lab5/pdf/154.pdf}
  \end{figure}
Se deshabilita el bloque del osciloscopio WX GUI y se habilita el
osciloscopio de constelaciones QT GUI, también se debe cambiar
en el bloque Options : Generate options : WX GUI por QT GUI
para que el bloque de constelaciones funcione.
\end{frame}

\begin{frame}{Modulación BPSK en GRC}
\begin{figure}
  \centering
   \includegraphics[width=0.8\textwidth]{lab5/pdf/155.pdf}
  \end{figure}
  \centering
Constelación BPSK
\end{frame}